\chapter{Simple Linear Regression}


The primary goal in regression is to a devlop a model that relates a set of explanatory variables $X_1, \ldots, X_p$ to a response variable $Y$, then test the model and use it for inference and predicition. 

Given a set of $n$ pairs of data $Y_i$ and $X_i$, we attempt to fit a straight line to these points, using a simple regression model 
\[Y_i = \beta_0 + \beta_1X_i + \epsilon_i\]

Where $\epsilon_i$ represents an unobserved random error term, $\beta_0$ is the intercept and $\beta_1$ is the slope of the line. $\beta_0$ and $\beta_1$ are parameters that need to be estimated from observed data. The model can also be expressed in terms of $(X_i - \bar{X})$.

\[Y_i = (\beta_0 + \beta_1\bar{X}) + \beta_1(X_i - \bar{X}) + \epsilon_i\]

Where $\bar{X}$ is the sample mean 
\[\bar{X} = \frac{1}{n}\sum_{i=1}^n  X_i \]

This proposed model is linear in the parameters $\beta_0$, $\beta_1$, and would still be referred to as linear if we had $X_i^2$ instead of $X_i$. This model also makes the assumption that the random error terms $\epsilon_i$ are uncorrelated, have mean 0, and variance $\sigma^2$. Under these assumptions, we have 
\[E(Y_i) = \beta_0 + \beta_1 X_i\]
\[\Var(Y_i) = \Var(\beta_0 + \beta_1X_i + \epsilon_i) = \sigma^2\]

Thus the mean of $Y$ is a linear function of $X$ however the variance of $Y$ does not depend on a value of $X$. 

The parameters $\beta_0$ and $\beta_1$ are called the regression coefficients. The slope $\beta_1$ is the change in the mean of the distribution of $Y$ produced by a unit change in $Y$. If the range of data on $X$ includes x = 0, then the intercept $\beta_0$ is the mean of the distribution of the response Y when x = 0. If the range of x does not include zero, then $\beta_0$ has no practical interpretation.

\section{Estimating the Parameters with the Method of Least Squares}

The parameters $\beta_0$, $\beta_1$ are unknown and must be estimated from the data.Suppose we have $n$ pairs of data $(x_1, y_1), (x_2,y_2), \ldots, (x_n,y_n)$.

\subsection{Estimation of $\beta_0$ and $\beta_1$}
The method of least squares is the most popular approach to fitting a regression model. Let $Q$ be the sum of the error terms squared 
\[Q = \sum_{i=1}^n \epsilon_i^2 = \sum_{i=1}^n (Y_i - \beta_0 - \beta_iX_i)^2\]

Then we want to minimize $Q$ with respect to the parameters $\beta_1$, $\beta_2$, 
\[\frac{\partial Q}{\partial \beta_0} = -2\sum_{i=1}^n (Y_i - \beta_0 - \beta_1X_i) = 0\]

\[\frac{\partial Q}{\partial \beta_1} = -2\sum_{i=1}^n (Y_i - \beta_0 - \beta_1X_i)X_i = 0\]    

We can rearrange these equations to get the following equations 

\begin{align*}
    &-2\sum_{i=1}^n (Y_i - \beta_0 - \beta_1X_i) = 0\\
    \implies& \sum_{i=1}^n Y_i - \sum_{i=1}^n\beta_0 - \sum_{i=1}^n\beta_1X_i = 0 \\
    \implies&  \sum_{i=1}^n Y_i = n\beta_0 - \beta_1\sum_{i=1}^nX_i \\
    &-2\sum_{i=1}^n (Y_i - \beta_0 - \beta_1X_i)X_i = 0\\
    \implies& \sum_{i=1}^n Y_iX_i - \sum_{i=1}^n\beta_0 X_i -  \sum_{i=1}^n\beta_1X_i^2 = 0\\
    \implies& \sum_{i=1}^n Y_iX_i = \beta_0\sum_{i=1}^nX_i - \beta_1 \sum_{i=1}^nX_i^2
\end{align*}
These 2 equations are known as the normal equations and the solutions to them, call them $b_0$, $b_1$, are
\[b_0 = \bar{Y} - b_1\bar{X}\]
\[b_1 = \frac{\sum_{i=1}^n (X_i - \bar{X})(Y_i - \bar{Y})}{\sum_{i=1}^n (X_i - \bar{X})^2 } = \frac{\sum_{i=1}^n(X_i -\bar{X})Y_i}{\sum_{i=1}^n(X_i-\bar{X})^2} = \sum_{i=1}^n k_iY_i\]
with 
\[k_i = \frac{X_i - \bar{X}}{\sum_{i=1}^n (X_i - \bar{X})^2}\]
We also sometimes use a more compact notation, by denoting the corrected sum of squares for $X$ and the sum of cross products of $X_i$ $Y_i$ as 
\[S_{xx} = \sum_{i=1}^n x_i - \frac{1}{n} \left(\sum_{i=1}^n x_i\right)^2 = \sum_{i=1}^n (x_i - \bar{x})^2 \]
\[S_{xy} = \sum_{i=1}^n x_iy_i - \frac{1}{n}\left(\sum_{i=1}^n x_i\right)\left(\sum_{i=1}^n y_i\right) = \sum_{i=1}^n y_i(x_i - \bar{x})\]
So, we can write 
\[b_1 = \frac{S_{xy}}{S_{xx}}\]
The observed difference between $Y_i$ and the corresponding fitted value $\hat{Y}_i$ is a residual. The $i$th residual is 
\[e_i = y_i - \hat{y}_i = y_i - (b_0 + b_1x_i)\]
Note that $k_i$ has important properites, such as 
\[\sum_{i=1}^n k_i = 0, \ \sum_{i=1}^n k_iX_i = 1,\ \sum_{i=1}^n k_i^2 = \frac{1}{\sum_{i=1}^n (X_i - \bar{X})^2}\]
\begin{align*}
    \sum_{i=1}^n k_i &= \frac{\sum_{i=1}^n (X_i - \bar{X})}{\sum_{i=1}^n (X_i - \bar{X})^2}\\
    &= \frac{n\bar{X} - n\bar{X}}{\sum_{i=1}^n (X_i-\bar{X})^2 } = 0
\end{align*}
\begin{align*}
    \sum_{i=1}^n k_iX_i &= \frac{\sum_{i=1}^n (X_i - \bar{X})X_i}{\sum_{i=1}^n (X_i - \bar{X})^2}\\
    &= \frac{\sum_{i=1}^n (X_i^2 - X_i\bar{X})}{\sum_{i=1}^n  (X_i - \bar{X})^2}\\
    &= \frac{\sum_{i=1}^n X_i^2 - \bar{X}\sum_{i=1}^n X_i}{\sum_{i=1}^n (X_i - \bar{X})^2}\\
    &= \frac{\sum_{i=1}^n X_i^2 - n\bar{X}^2}{\sum_{i=1}^n (X_i^2 - 2X_i\bar{X} + \bar{X}^2)}\\
    &= \frac{\sum_{i=1}^n X_i^2 - n\bar{X}^2}{\sum_{i=1}^n X_i^2 - 2n\bar{X}^2 + n\bar{X}^2} = 1
\end{align*}
\begin{align*}
    \sum_{i=1}^n k_i^2 &= \sum_{i=1}^n \left(\frac{X_i-\bar{X}}{\sum_{i=1}^n (X_i - \bar{X})^2}\right)^2\\
    &= \frac{\sum_{i=1}^n (X_i-\bar{X})^2}{\sum_{i=1}^n (X_i-\bar{X})^4}\\
    &= \frac{1}{\sum_{i=1}^n (X_i-\bar{X})^2}
\end{align*}
\noindent
The equation for the fitted line is then 
\[\hat{Y} = b_0 + b_1X\]
Or alternatively using $X - \bar{X}$, 
\[\hat{Y} = (b_0 + b_1\bar{X}) + b_1(X - \bar{X})\]

\begin{theorem}[Gauss Markov Theorem]
    The least square estimators $b_0$, $b_1$ are unbiased and have minimum variance among all unbiased linear estimators.  
\end{theorem}

\begin{proof}
    Consider an unbiased linear estimator 
    \[\hat{\beta}_1 = \sum_{i=1}^n c_iY_i\]
    $\hat{\beta}_1$ must satisfy $E(\hat{\beta_1}) = \beta_1$.
    \begin{align*}
        \beta_1 &= E(\hat{\beta}_1)\\
        &= E\left(\sum_{i=1}^n c_i Y_i\right)\\
        &= \sum_{i=1}^n c_i E(Y_i)\\
        &= \sum_{i=1}^n c_i(\beta_0 + \beta_1 X_i)\\
        &= \beta_0\sum_{i=1}^n c_i + \beta_1\sum_{i=1}^n c_iX_i  
    \end{align*}
    Therefore, $\sum\limits_{i=1}^n c_i = 0$, and $\sum\limits_{i=1}^n c_iX_i = 1$. We can also see that the variance is 
    \[\Var(\hat{\beta_1}) = \sum_{i=1}^n c_i^2 \Var(Y_i) = \sigma^2\sum_{i=1}^n c_i^2\]
    Now, set $c_i = k_i + d_i$ where $k_i$ is as defined previously above and $d_i$ are arbitrary constants. We want to show that the variance is minimized, so 
    \begin{align*}
        \Var(\hat{\beta}_1) &= \sum_{i=1}^n c_i^2\Var(Y_i)\\
        &= \sigma^2\sum_{i=1}^n c_i^2\\
        &= \sigma^2\sum_{i=1}^n (k_i + d_i)^2\\
        &= \sigma^2 \left(\sum_{i=1}^n k_i^2 + 2\sum_{i=1}^n k_id_i + \sum_{i=1}^n d_i^2\right)
    \end{align*}
    Note that the variance of $b_1$ is 
    \[\Var(b_1) = \Var\left(\sum_{i=1}^n k_iY_i\right) = \sigma^2 \sum_{i=1}^n k_i^2 = \sigma^2 \frac{1}{\sum_{i=1}^n (X_i - \bar{X})^2}\]
    Now notice that there is a relationship between the variance of $\hat{\beta}_1$ and $b_1$, namely that the variance of $\hat{\beta}_1$ is the same as $b_1$ plus an additonal constants but these constants are indeed 0. 
    \begin{align*}
        \sum_{i=1}^n k_id_i &= \sum_{i=1}^n k_i(c_i - k_i)\\
        &= \sum_{i=1}^n k_ic_i - \sum_{i=1}^n k_i^2\\
        &= \sum_{i=1}^n c_i \frac{X_i - \bar{X}}{\sum_{i=1}^n (X_i - \bar{X})^2} - \frac{1}{\sum_{i=1}^n (X_i - \bar{X})^2}\\
        &= \frac{\sum_{i=1}^n c_iX_i  - \sum_{i=1}^n c_i\bar{X}}{\sum_{i=1}^n (X_i - \bar{X})^2} - \frac{1}{\sum_{i=1}^n (X_i - \bar{X})^2}\\
    \end{align*}
    We know that $\sum_{i=1}^n c_i = 0$ and $\sum_{i=1}^n c_i X_i = 1$, so this becomes
    \[\sum_{i=1}^n k_id_i = \frac{1 - 0}{\sum_{i=1}^n (X_i - \bar{X})^2} - \frac{1}{\sum_{i=1}^n (X_i - \bar{X})^2} = 0\]
    Therefore, 
    \[\Var(\hat{\beta}_1) = \sigma^2 \left(\sum_{i=1}^n k_i^2 + \sum_{i=1}^n d_i^2\right)\]
    Clearly the variance is minimized when $d_i = 0$ for all $i$, thus 
    \[\Var(\hat{\beta_1}) = \sigma^2 \sum_{i=1}^n k_i^2 = \Var(b_1)\]
    Thus the least squares estimator $b_1$ has minimum variance along all unbiased estimators.
\end{proof}

\noindent
We may write 
\[\hat{Y} = b_0 + b_1X\]
for the estimated or fitted line, and 
\[e_i = Y_i - \hat{Y}_i\]
for the estimated $i^{th}$ residual. The estimate for the variance $\sigma^2$ is then 
\[\hat{\sigma}^2 = \frac{\sum_{i=1}^n e_i^2}{n-2}\]
The estimate of the variance $\sigma^2$ is also known as the mean square error (MSE). 

\subsection{Properties of Fitted Regression Line}
\begin{enumerate}[label=(\roman*)]
    \item $\sum\limits_{i=1}^n e_i = 0$. Recall that $\hat{Y} = b_0 + b_1X = (b_0 + b_1\bar{X}) + b_1(X - \bar{X})$, and 
    \[\bar{Y} = b_0 + b_1\bar{X}\]
    So $\hat{Y} = \bar{Y} + b_1(X - \bar{X})$, then
    \begin{align*}
        \sum_{i=1}^n e_i &= \sum_{i=1}^n (Y_i - \hat{Y}_i)\\
        &= \sum_{i=1}^n Y_i - \sum_{i=1}^n \hat{Y}_i\\
        &= \sum_{i=1}^n Y_i - \sum_{i=1}^n (\bar{Y} + b_1(X_i - \bar{X}))\\
        &= n\bar{Y} - n\bar{Y} + b_1\sum_{i=1}^n (X_1 - \bar{X})\\
        &= n\bar{Y} - n\bar{Y} + b_1(n\bar{X} - n\bar{X}) = 0
    \end{align*}
    \item $\sum\limits_{i=1}^n Y_i = \sum\limits_{i=1}^n \hat{Y}_i$. This follows from the previous property since 
    \[\sum_{i=1}^n e_i = \sum_{i=1}^n Y_i - \sum_{i=1}^n \hat{Y}_i = 0 \implies \sum_{i=1}^n Y_i = \sum_{i=1}^n \hat{Y}_i\]
    \item $\sum\limits_{i=1}^n X_ie_i = 0$. This can be shown from the definition
    \begin{align*}
        \sum_{i=1}^n X_ie_i &= \sum_{i=1}^n X_i(Y_i - \hat{Y}_i)\\
        &= \sum_{i=1}^n X_i(Y_i - b_0 - b_1X_i)\\
        &= \sum_{i=1}^n X_iY_i - b_0\sum_{i=1}^n X_i - b_1\sum_{i=1}^nX_i^2\\
        &= b_0 \sum_{i=1}^n X_i + b_1\sum_{i=1}^n X_i - b_0 \sum_{i=1}^n X_i - b_1\sum_{i=1}^n X_i\\
        &= 0
    \end{align*}
    This is signficant because it tells us that the dot product between the vector of explanatory variables $\vec{X} = (X_1, \ldots, X_i)^T$ is orthogonal to the vector of error terms $\vec{e} = (e_1, \ldots, e_n)^T$, and from the previous property we get that 
    \[\vec{e} \cdot 1_n = \sum_{i=1}^n e_i = 0\]
    Hence the vectors $\{1_n, X - \bar{X}1_n\}$ are linearly independent and form a basis of the estimation space. 
    \item By applying the Pythagorean Theorem to the previous property we get 
    \begin{align*}
        ||Y||^2 &= ||\hat{Y}||^2 + ||Y-\hat{Y}||^2\\
        \sum_{i=1}^n Y_i^2 &= \sum_{i=1}^n \hat{Y}_i^2 + \sum_{i=1}^n e_i^2\\
        &= \sum_{i=1}^n \bar{Y}^2 + b_1^2\sum_{i=1}^n (X_i - \bar{X})^2 + \sum_{i=1}^n e_i^2\\
        \implies \sum_{i=1}^n Y_i^2 - n\bar{Y}^2 &= b_1^2\sum_{i=1}^n (X_i -\bar{X})^2 + \sum_{i=1}^n e_i^2\\
        \sum_{i=1}^n (Y_i - \bar{Y})^2 &= b_1^2\sum_{i=1}^n (X_i - \bar{X})^2 + \sum_{i=1}^n \left(Y_i - \hat{Y}_i\right)^2     
    \end{align*}
    This shows us the the total sum of squares is equal to the regression sum of squares plus the error sum of squares. 
    \item The point $(\bar{X}, \bar{Y})$ is on the fitted line. 
    \item The sum of residuals weighted by their corresponding fitted value is 0, that is 
    \[\sum_{i=1}^n y_ie_i = 0\]
    \item Under the normality assumption, $e_i \iid N(0,\sigma^2)$. The method of maximum likelihood leads to the method of least squares.
    \[L(\beta_0, \beta_1,\sigma^2) = \left(\frac{1}{\sqrt{2\pi}\sigma}\right)^n\exp\left(-\frac{1}{2\sigma^2}\sum_{i=1}^n \epsilon_i^2\right)\]
    So maximizing $L(\beta_0,\beta_1, \sigma^2)$ is equivalent to minimizing $\sum \epsilon_i^2$.
\end{enumerate}

 \subsection{Estimation of $\sigma^2$}

 We need to estimate $\sigma^2$ to test hypotheses and construct interval estimates pertinent to the regression model. Ideally we would like this estimate not to depend on the adequacy of the fitted model. This is only possible when there are several observations on $y$ for at least one value of $x$, or when prior information concerning $\sigma^2$ is available. When this approach cannot be used, the estimate of $\sigma^2$ is obtained from the residual or error sum of squares.

\[SSE = \sum_{i=1}^n e_i^2 = \sum_{i=1}^n (y_i - \hat{y}_i)^2\]
We can substitute $\hat{y}_i$ for $b_0 + b_1x_i$ and simplify to get 
\[SSE = \sum_{i=1}^n y_i^2 - n\bar{y}^2 - b_1S_{xy}\]
Morever, the correct sum of squares of the response variable is 
\[SST = \sum_{i=1}^n y_i^2 - n\bar{y}^2 = \sum_{i=1}^n (y_i - \bar{y})^2\]
Thus,
\[SSE = SST  - b_1S_{xy}\]

The residual sum of squares has $n-2$ degrees of freedom, because we reserve 2 degrees of freedom for the estimators $b_0$, $b_1$. We will later show that the expected value for $SSE$ is 
\[E(SSE) = (n-2)\sigma^2 \]
So an unbiased estimator of $\sigma^2$ is 
\[\hat{\sigma}^2 = \frac{SSE}{n-2} = MSR\]

The quantity $MSR$ is known as the \textbf{residual mean square}. The root of $\hat{\sigma}^2$ is known as the \textbf{standard error of regression.}

\section{Hypothesis Testing on the Slope and Intercept}

To preform hypotheses tests and construct confidence intervals, we require that we make the additional assumption that the model errors $\epsilon_i$ are normally distributed. Thus, the complete assumptions are that the errors are normally and independently distributed with mean 0 and variance $\sigma^2$, written as $\{\epsilon_i\} \iid N(0, \sigma^2)$. We will discuss how these assumptions can be checked through residual analysis later.

Suppose that we have the model $Y_i = \beta_0 + \beta_1X_i + \epsilon_1$, where $\{\epsilon_i\} \iid N(0,\sigma^2)$. Then 
\begin{enumerate}[label=(\alph*)]
    \item $\frac{b_1 - \beta_1}{se(b_1)} \sim t_{n-2}$ where $se^2(b_1) = \frac{MSE}{\sum_{i=1}^n (X_i - \bar{X})^2}$
    \item $\frac{b_0 - \beta_0}{se(b_0)} \sim t_{n-2}$ where 
    \[se^2(b_0) = MSE\left(\frac{1}{n} + \frac{\bar{X}^2}{\sum_{i=1}^n (X_i-\bar{X})^2}\right)\]
    \item MSE is an unbiased estimate of $\sigma^2$ and is independent of $b_0$,$b_1$. Furthermore
    \[\frac{(n-2)MSE}{\sigma^2} \sim \chi_{n-2}^2\]
\end{enumerate}
\begin{proof}
    Proof will be shown when we generalize this using matrices in later sections.
\end{proof}

\subsection{Using $t$-tests}

Suppose we want to test that the slope is equal to a constant, $\beta$, we have the hypotheses 
\[H_0: \beta_1 = \beta, \ H_1: \beta_1 \neq \beta\]
Since $\{\epsilon_i\} \iid N(0,\sigma^2)$, the observations $y_i$ are normally distributed with $\beta_0  + \beta_1x_i$ and variance $\sigma^2$. Then, $b_1$ is a linear combination of the observations, so it is normally distributed with mean $\beta_1$ and variance $\sigma^2/S_{xx}$. Therefore, our test statistic becomes 
\[Z_0 = \frac{b_1 - \beta}{\sqrt{\sigma^2/S_{xx}}}\]
If the null hypothesis is true, then $Z_0 \sim N(0,1)$. If $\sigma^2$ is known then we would use $Z_0$ to test our hypotheses. However, $\sigma^2$ is typically unknown. We've seen that $MSE$ is an unbiased estimator for $\sigma^2$, and we've established that $(n-2)MSE/\sigma^2 \sim \chi_{n-2}^2$.

\[t_0 = \frac{b_1 - \beta}{\sqrt{MSE/S_{xx}}}\]

If the null hypothesis is true, $t_0 \sim t_{n-2}$. We compare the observed value $t_0$ with the upper $\alpha/2$, of the $t_{n-2}$ distribution. So we reject the null hypothesis

\[|t_0| > t_{\alpha/2,n-2}\]

We can also test with the $p$-value. From the equation for $t_0$, the denominator is called the \textbf{estimated standard error} of the slope.  

\[se(b_1) = \sqrt{\frac{MSE}{S_{xx}}}\]

So, we often write $t_0$ is 

\[t_0 = \frac{b_1 - \beta}{se(\beta_1)}\]

We test the intercept in a similar manner, 
\[H_0: \beta_0 = \beta, \ H_1: \beta_0 \neq \beta\]
We use a similar test statistic,
\[t_0 = \frac{b_0 - \beta}{se(b_0)}\]
and we reject the null hypothesis when $|t_0| > t_{\alpha/2, n-2}$.

\subsection{Testing Significance}

A special case for hypotheses is 
\[H_0: \beta_1 = 0, \ H_1: \beta_1 \neq 0\]

These hypotheses relate to the \textbf{significance of regression}. Failing to reject the null hypothesis means there is no linear relationship between $x$ and $y$, we would reject the null hypothesis when $|t_0| > t_{\alpha/2, n-2}$. 

\subsection{Analysis of Variance Tables (ANOVA)}

\textbf{Analysis of variance} can be used to test significance of regression. The analysis of variance of variance is based on a partitioning of the total variability of the response variable $y$, given by 
\[y_i - \bar{y} = (\hat{y}_i - \bar{y}) + (y_i - \hat{y}_i)\]
Then, taking the sum of the square of both sides
\[\sum_{i=1}^n (y_i - \bar{y})^2 = \sum_{i=1}^n(\hat{y}_i - \bar{y})^2 + \sum_{i=1}^n(y_i - \hat{y}_i)^2 + 2\sum_{i=1}^n(\hat{y}_i - \bar{y})(y_i - \hat{y}_i)\]
Notice that 
\begin{align*}
    2\sum_{i=1}^n(\hat{y}_i - \bar{y})(y_i - \hat{y}_i) &= 2\sum_{i=1}^n \hat{y}_i(y_i - \hat{y}_i) - 2 \bar{y}\sum_{i=1}^n(y_i - \hat{y_i})\\
    &= 2\sum_{i=1}^n \hat{y}_i e_i - 2\bar{y}\sum_{i=1}^n e_i = 0\\
\end{align*}

Therefore, 

\[\sum_{i=1}^n (y_i - \bar{y})^2 = \sum_{i=1}^n (\hat{y}_i - \bar{y})^2 + \sum_{i=1}^n (y_i - \hat{y}_i)^2\]

The left side is the corrected sum of squares of the observations, which we denote by $SST$ or $SSTO$. Notice that $y_i - \hat{y}_i = e_i$, so that term is the sum of residuals squared $SSE$. We call $\sum\limits_{i=1}^n (\hat{y}_i - \bar{y})^2$ the \textbf{regression sum of squares}. So we have 
\[SST =  SSR + SSE\]
The regression sum of squares can also be computed by 
\[SSR = b_1^2S_{xx}\]

The \textbf{degrees of freedom} for each sum of squares is as follows. 

\begin{itemize}
    \item The total sum of squares $SST$ has $df_T = n-1$ since we lose a degree of freedom for the constraint 
    \[\sum_{i=1}^n (y_i - \bar{y}) = 0\]
    \item The regression sum of squares $SSR$ has $df_R = p-1$ where $p$ is the number of variables (including $y$).
    \item The residual sum of squares $SSE$ has $df_E = n-2$ degrees of freedom since 2 constraints are placed on $e_i = y_i - \hat{y}_i$ with the estimation for $beta_0$ and $\beta_1$.
\end{itemize}


We create a table to summarize our results from statistical analysis. 

\begin{center}
    \begin{tabular}{|c|c|c|c|c|}
        \hline
        Source & SS & DF & MS=SS/df & E(MS)\\
        \hline
        \hline
        Regression & $SSR = b_1^2\sum(X_i-\bar{X})^2$ & $p-1$ & MSR & $\sigma^2 + \beta_1^2\sum(X_i - \bar{X})^2$\\
        \hline
        Error & $SSE=\sum(Y_i - \hat{Y}_i)^2$ & $n-p$ & MSE & $\sigma^2$\\
        \hline 
        Total & $SSTO =\sum(Y_i - \bar{Y})^2$ & $n-1$ & & \\
        \hline
    \end{tabular}
\end{center}

Each of the sums of squares is a quadratic form where the rank of the corresponding matrix is the degrees of freedom indicated. Chochran's theorem applies and we conclude that the quadratic forms are independent and have chi-sqaured distributions. Note that 
\[\frac{SSR}{\sigma^2} = \frac{b_1^2\sum(X_i - \bar{X})^2}{\sigma^2} \sim \chi^2_{p-1}\]
\[\frac{SSE}{\sigma^2} = \frac{\sum (Y_i - \hat{Y}_i)^2}{\sigma^2} \sim \chi^2_{n-p}\]
Then, the ratio between 2 chi-sqaured distributions divided by their degrees of freedom has a F-distribution with their respective degrees of freedom. 
\[F = \frac{SSR/\sigma^2(p-1)}{SSE/\sigma^2(n-p)} = \frac{SSR/(p-1)}{SSE/(n-p)} = \frac{MSR}{MSE} \sim F_{p-1,n-p}\]

The degrees of freedom are determined by the amount of data required to calculate each expression. \\
To summarize, the ANOVA table indactes how one can test the null hypothesis
\[H_0 : \beta_1 = 0\]
\[H_1 : \beta_1 \neq 0\]
The null Hypothesis is that the slope of the line is equal to 0. Under the null hypothesis, the expected mean square for regression and the expected mean square error are seperate independent estimates of the variance $\sigma^2$.


% \begin{itemize}
%     \item $\sum\limits_{i=1}^n (Y_i-\bar{Y})^2$ has $n-1$ degrees of freedom since we have 1 constraint on the data that 
%     \[\sum_{i=1}^n (Y_i - \bar{Y}) = 0\]
%     \item $b_1^2\sum\limits_{i=1}^n (X_i - \bar{X})^2$ has one degree of freedom because it is a function of $b_1$
%     \item $\sum\limits_{i=1}^n(Y-i - \hat{Y}_i)^2$ has $n-2$ degrees of freedom because it is a function of 2 parameters
% \end{itemize}

\section{Interval Estimation}

\subsection{Confidence Intervals on $\beta_0$, $\beta_1$, and $\sigma^2$.}

The width of the confidence intervals are a measure of the quality of the regression line. If the error is normally and independently distributed by our assumption, then $(b_1 - \beta_1)/se(b_1)$ and $(b_0 - \beta_0)/se(b_0)$ follow a $t$ distribution with $n-2$ degrees of freedom. So, a $100(1 - \alpha)$ percent. The confidence interval for the slope $\beta_1$ is 
\[b_1 - t_{\alpha/2, n-2}se(b_1) \leq b_1 \leq b_1 + t_{\alpha/2, n-2}se(b_1)  \]
and for the intercept $\beta_0$, 
\[b_0 - t_{\alpha/2, n-2}se(b_0) \leq b_0 \leq b_0 + t_{\alpha/2, n-2}se(b_0)\]


The interpretation for these intervals is, if we were to take repeated samples of the same size at the same x levels and construct 95\% CIs on the slope for each sample, then 95\% of those intervals will contain the true value of $\beta_1$.

As we've seen earlier, the sampling distribution of $(n-2)MSE/\sigma^2$ follows a chi-sqaure distribution with $n-2$ degrees of freedom, so 
\[P\left\{\chi^2_{1-\alpha/2, n-2} \leq \frac{(n-2)MSE}{\sigma^2} \leq \chi^2_{\alpha/2, n-2}\right\}\]
Thus the $100(1-\alpha)$ percent CI on $\sigma^2$ is 
\[\frac{(n-2)MSE}{\chi^2_{\alpha/2, n-2}} \leq \sigma^2 \leq \frac{(n-2)MSE}{\chi^2_{1 - \alpha/2, n-2}}\]

\subsection{Interval Estimation of the mean Response}

Another important part of the regression model is estimating the mean response $E(y)$ for a particular regressor variable $x$. Assuming that $x_0$ is any value of the regressor variable within the range of the original data on $x$ that we used to create the moel. Then, an unbiased estimator for $E(y|x_0)$ can be found from the fitting model 
\[\widehat{E(y|x_0)} = \hat{\mu}_{y|x_0} =b_0 + b_1x_0\]

Note that $\hat{\mu}_{y|x_0}$ follows a normal distribution since it is a linear combination of the observations $y_i$. The variance is 
\[\Var(\hat{\mu}_{y|x_0}) = \Var(b_0 + b_1x_0) = \Var(\bar{y} - b_1(x_0 -\bar{x})) = \frac{\sigma^2}{n} + \frac{\sigma^2(x_0-\bar{x})^2}{S_{xx}}\]

The sampling distribution for 
\[\frac{\hat{\mu}_{y|x_0} - E(y|x_0)}{\sqrt{MSE(1/n + (x_0-\bar{x})^2/S_{xx})}}\]
is a $t$ distribution with $n-2$ degrees of freedom. Then the CI is given as 
\[\left[\hat{\mu}_{y|x_0} \pm t_{\alpha/2, n-2}\sqrt{MSE\left(\frac{1}{n} + \frac{(x_0 - \bar{x})^2}{S_{xx}}\right)} \right]\]


\section{Prediction of New Observations}

When we want to predict a new value for our regressor variable, say $x=x_0$, we obtain a point estimate for the response $y$, given as 
\[\hat{y}_0 = \hat{\beta_0} + \hat{\beta_1}x_0\]
Now we want to obtain an interval estimate for our new observation $y_0$ and conduct hypothesis tests. Note that the confidence interval for the mean response at $x=x_0$ is \textbf{not} the same. We define the new random error variable 
\[\psi = y_0 - \hat{y}_0\]
which is normally distributed with mean zero and variance 
\[\Var(\psi) = \Var(y_0 - \hat{y}_0) = \sigma^2\left(1+\frac{1}{n} + \frac{(x_0-\bar{x})^2}{S_{xx}}\right)\]
Then, we use the standard error for $\psi$ for our prediction interval, giving us the $100(1-\alpha)$ percent prediction interval 
\[\left[\hat{y}_0 \pm t_{\alpha/2, n-2} \sqrt{MSE\left(1+\frac{1}{n} + \frac{(x_0-\bar{x})^2}{S_{xx}}\right)}\right]\]
We can also conduct hypothesis testing, suppose our hypotheses are $H_0: y_0 = y_{00}$, $H_1: y_0 \neq y_{00}$, then we use the test statistic
\[\frac{\hat{y}_0 - y_{00}}{\sqrt{MSE\left(1+\frac{1}{n} + \frac{(x_0-\bar{x})^2}{S_{xx}}\right)}} = \frac{\hat{y}_0 - y_{00}}{se(\psi)}  \sim t_{n-2}\]
We reject the null hypothesis when $|t_0| > t_{\alpha/2, n-2}$.



\section{Coefficient of Determination}

The quantity 
\[R^2 = \frac{SSR}{SST} = 1 - \frac{SSE}{SST}\]

is called the \textbf{coefficient of determination.} $R^2$ is also called the proportion of variation explained by the regressor $x$ since $SST$ is a measure of variability in $y$ without considering the effect of $x$, and $SSE$ is the variability in $y$ after considering $x$. Since $0 \leq SSE \leq SST$, then $0 \leq R^2 \leq 1$. An $R^2$ value close to 1 means \textbf{most of the variability of $y$ is explained by $x$.}  


\section{Correlation Coefficient}

The pearson correlation coefficient, denoted by $\rho$, related to $b_1$ is given as
\[\rho = \frac{\Cov(X,Y)}{\sqrt{\Var(X)}\sqrt{\Var(Y)}}\]
This measures the linear correlation between 2 variables. When applied to a sample, 
\[r = b_1\left(\frac{S_{xx}}{SST}\right)^{\frac{1}{2}} = \frac{\sum_{i=1}^n(X_i - \bar{X})(Y_i - \bar{Y})}{\sqrt{\sum_{i=1}^n (X_i - \bar{X})^2(Y_i - \bar{Y})^2}} = \frac{S_{xy}}{(S_{xx}SST)^{1/2}}\]
Note that $-1 \leq r \leq 1$. To test hypotheses on $\rho$, we have 2 cases. The hypotheses for testing if the correlation is 0 is as follows 
\[H_0: \rho = 0, \ H_1: \rho \neq 0\]
When testing the null hypothesis $\rho = 0$, we use a $t$ statistic given as 
\[t = \frac{r\sqrt{n-2}}{\sqrt{1-r^2}} \sim t_{n-2}\]
When testing 
\[H_0: \rho = \rho_0, \ H_1: \rho \neq \rho_0\]
We use a $Z$ statistic, 
\[Z = \atanh r = \frac{1}{2}\ln \left(\frac{1+r}{1-r}\right) \sim N\left(\mu_z, \frac{1}{n-3}\right)\]
where 
\[\mu_z = \frac{1}{2}\ln\left(\frac{1+\rho}{1 - \rho}\right)\]
Now we can standardize our statistic to obtain a standard normal test statistic
\[Z_0 = (\atanh(r) - \atanh(\rho_0))\sqrt{n-3}\]
We can obtain our confidence interval with 
\[\tanh\left(\atanh(r) - \frac{Z_{\alpha/2}}{\sqrt{n-3}}\right) \leq \rho \leq \tanh\left(\atanh(r) + \frac{Z_{\alpha/2}}{\sqrt{n-3}}\right)\]
where $\tanh(u) = (e^u-e^{-u})/(e^u + e^{-u})$.