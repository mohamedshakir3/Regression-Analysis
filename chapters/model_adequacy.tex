\chapter{Model Adequacy}

We've made various assumptions thus far, such as 
\begin{enumerate}
    \item The relationship between the response $y$ and the regressors is linear
    \item The error terms $\epsilon$ are normally distributed with mean 0 and variance $\sigma^2$.
    \item The errors are uncorrelated.
\end{enumerate}
\noindent
We want to ceheck the validity of these assumptions.
\section{Residual Analysis}
Recall the definition for our observed residuals 
\[e_i = y_i - \hat{y}_i\]
Their variance is estimated by 
\[\frac{1}{n-p}\sum_{i=1}^n (y_i - \hat{y}_i)^2 = \frac{SSE}{n-p} = MSE\]
The residuals are not independent, however, as the $n$ residuals have only $n - p$
degrees of freedom associated with them. This nonindependence of the residuals
has little effect on their use for model adequacy checking as long as $n$ is not small
relative to the number of parameters $p$.

\subsection{Checking for Normality}

If the assumption of normality holds, a box plot of the residuals should indicate a symmetric box around the median of 0. A histogram of the residuals can also be used to examine normality. If the residuals follow a similar curve as a normal distribution, then it suggests that the normaility assumption is reasonable. The skewness and kurtosis can also help provide insight in this case, a normal distribution has a skewness of 0 and a kurtosis of 3. Finally, a quantile-quantile plot (also known as a qq-plot) comapres the quantiles of the residual data with the quantiles from a normal distribution. We compute 
\[E_k = \sqrt{MSE}\cdot\Phi^{-1}\left(\frac{k-0.375}{n + 0.25}\right), \ k = 1, \ldots , n\]
Where $\Phi$ is the standard normal, and we plot $e_{(k)}$ vs $E_k$ where $e_{(k)}$ is the residual with rank $k$. Under normality, we would expect a straight line.

\subsection{Checking Constant Variance}

We use $MSE$ as an estimate for approximating the variance of residual. We can improve the residual scaling by dividing the residuals $e_i$ by the exact standard deviations for the $i$th residual. Recall that we can use the notion with the hat matrix $H$ to write the vector of residuals as 
\[\boldsymbol{e} = (I-H)Y\]
where $H = X(X'X)^{-1}X'$ is the hat matrix. Recall the properties of the hat matrix, namely that it is symmetric and idempotent, and $I-H$ has the same properties. We can write the residuals as 
\[\boldsymbol{e} = (I-H)(X\boldsymbol{\beta} + \boldsymbol{\epsilon}) = (I-H)\boldsymbol{\epsilon}\]
The covariance matrix of the residuals is 
\[\Var(\boldsymbol{e}) = \Var[(I-H)\boldsymbol{\epsilon}] = (I-H)\Var(\boldsymbol{\epsilon})(I-H)' = \sigma^2(I-H)\]
This gives us the variance of the $i$th residual as
\[\Var(e_i) = \sigma^2(1-h_{ii})\]
and the covariance between residuals $e_i$ and $e_j$ as 
\[\Cov(e_i,e_j) = -\sigma^2h_{ij}\]
Now, we can studentize the reisduals to obtain 
\[r_i = \frac{e_i}{\sqrt{MSE(1-h_{jj})}}\]
We can plot the studentized residuals vs fitted values to check for non-constancy of variance. The plot should show a random distribution of points. Conversly, non-constance variance would appear as a pattern, an increasing or decreasing collection of points. A scale-location plot can also be used to exam homogeneity, by plotting 
\[\sqrt{|r_i|} \text{ vs } \hat{Y}_i\]
If the residuals lie in a narrow band around 0 then there are no evidence to suggest we need corrections. Otherwise, if the residuals show a pattern, either increasing or decreasing, this is a sign that variance is non constant. If a double-bow pattern appears, this is indication that the variance in the middle is larger than the variance at the extremes. If the residuals appear to have a quadratic relationship (i.e a parabola shape), there may be a nonlinear relation that the model has not accounted for.